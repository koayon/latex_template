% This provides the narrative for the paper and most people won't make it past this section in terms of reading
% (they may skip to the pictures later though).

% Aim for a single page.

% Sections:
% What is the problem
% Why is it interesting and important (and to whom)?
% Why is it hard? Does the approach trivially work?
% Why hasn't it been solved before? How do we reframe the problem to solve it better?
% What are the key components of the approach? Explain these in a figure.

% If you want to say "much recent interest", "increasingly important" etc. its good to have recent citations to back this up.

% We want to get across:
% WHY IS IT NOVEL
% WHY IS IT IMPORTANT
% REMOVE EVERYTHING ELSE
% Don't beat around the bush; if the point is "A, therefore B" (where B is some good fact about your work), then say that, rather than being humble and just pointing out A.
% Make sure the importance of the contribution is clear.

% EXPLAINING THE PROBLEM, WHY IT MATTERS AND THE SOLUTION
% THE KEY FIGURE

% If you can write an amazing first sentence, this is often worth it.
% E.g. Barlow (1961): “A wing would be a most mystifying structure if one did not know that birds flew.”
% Start by detailing why you're interested in the problem not with some nebulous appeal to "the literature" which has been interested in the problem for a long time.

\begin{figure}[h]
  \centering
  \begin{minipage}{0.55\linewidth}
      \centering
      % If you want to frame the image, uncomment the next line and comment out the original \includegraphics line
      % \fbox{\includegraphics[width=\linewidth]{assets/chart_placeholder.png}}
      \includegraphics[width=\linewidth]{assets/chart_placeholder.png}
  \end{minipage}
\caption{Well-designed pull/hook figure on page 1 or 2 to bring the reader in and convey the central point of the paper}
\label{fig:hook_figure}
\end{figure}

% It's often nice to introduce a running example here which

\subsection{Summary of Contributions}
\label{contributionsummary}
% - tell the reader why they should care about a paper right away, possibly even doing it explicitly

Our contributions are as follows:

\begin{itemize}
%\tightlist
\item
  Contribution 1

\item
  Contribution 2

\end{itemize}

% IMPORTANCE: ...
% NOVELTY: ...

% The introduction is making claims.
% The body of the paper mainly serves to back up these claims with experimental evidence or theoretical arguments.
% Ideally here we convey the intuition of the paper, so that people can later follow the details. Simple is good.
% This is basically "however you'd explain this on a whiteboard (but with better grammar)".
% Remember your running example!
% If possible "example -> general problem -> solution".

% Take the most direct route to your idea, don't follow the path that you took to get there.


We open-source a reference implementation for the community at [REDACTED].

The remainder of this paper is organized as follows: In \textit{Section 2}, ...

% LINKS ABOVE to the rest of the paper section

% If you don't have someone's attention by now, you've lost them.
