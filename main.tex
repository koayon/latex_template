\documentclass{article} % For LaTeX2e
\usepackage{styles/iclr2025_conference,times}

% Optional math commands from https://github.com/goodfeli/dlbook_notation.
\input{styles/math_commands}

\usepackage[utf8]{inputenc} % allow utf-8 input
\usepackage[T1]{fontenc}    % use 8-bit T1 fonts
% \usepackage{hyperref}       % hyperlinks
% \usepackage{url}            % simple URL typesetting
\usepackage{booktabs}       % professional-quality tables
\usepackage{amsfonts}       % blackboard math symbols
\usepackage{nicefrac}       % compact symbols for 1/2, etc.
\usepackage{microtype}      % microtypography
\usepackage{graphicx}
\usepackage{natbib}
\usepackage{doi}
\usepackage{amssymb}
% Sandy added
% \usepackage{todonotes}
\usepackage{amsmath}
\usepackage{appendix}
\usepackage{algorithm}
\usepackage{algpseudocode}
\usepackage{import}
\usepackage{cleveref}
\usepackage{hyperref}
\usepackage{url}

\title{ENTER TITLE HERE}
% Title is important - it should be clear, snappy and informative.
% Make it SEO-friendly; include whatever people might search for when vaguely looking for your paper.

\author{{\hspace{1mm}Kola Ayonrinde} \\
        MATS \\
	\texttt{koayon@gmail.com} \\
        % \And
        % ...
}

% Authors must not appear in the submitted version. They should be hidden
% as long as the \iclrfinalcopy macro remains commented out below.
% Non-anonymous submissions will be rejected without review.

\import{paper_sections/}{macros.tex}

% Always pay attention to LaTeX warning lest they get out of hand later.

\iclrfinalcopy % Uncomment for camera-ready version, but NOT for submission.
\begin{document}


\maketitle
% \thispagestyle{fancy}

% use \emph{} for emphasis not \textit

% Every section of the paper should tell a story.
% The story should be linear, keeping the reader engaged at every step and looking forward to the next step.

% Suppose someone just saw the all the images and captions and nothing else - do they understand the gist?
% What about this plus the abstract? This is probably 80:20 people.
% If the answer is no, then update until this is true.

% INTRODUCTORY SECTIONS

\import{paper_sections/}{abstract.tex}

\section{Introduction}
\label{sec:intro}
\import{paper_sections/}{introduction.tex}

% EXPLANATORY SECTIONS

\section{Related Work}
\label{sec:related_work}
\import{paper_sections/}{related_work.tex}

\section{Background}
\label{sec:background}
\import{paper_sections/}{background.tex}

\section{Methods}
\label{sec:methods}
\import{paper_sections/}{methods.tex}

% EMPIRICAL SECTIONS

\hypertarget{setup}{%
\section{Experimental Setup}\label{setup}}
\import{paper_sections/}{experimental_setup.tex}

\section{Results}
\label{sec:results}
\import{paper_sections/}{results.tex}

\hypertarget{discussion}{%
\section{Discussion}\label{discussion}}
\import{paper_sections/}{discussion.tex}

% CONCLUDING SECTIONS

\hypertarget{conclusion}{%
\section{Conclusion}\label{conclusion}}
\import{paper_sections/}{conclusion.tex}

% \hypertarget{acknowledgments}{%
% \subsubsection*{Acknowledgments}\label{acknowledgments}}
% \import{paper_sections/}{acknowledgements.tex}

\newpage

\bibliography{references}
\bibliographystyle{styles/iclr2025_conference}

% APPENDICES

\newpage
\appendix % This command starts the appendix section
% \appendixpage

\hypertarget{appendix1}{%
\section{Appendix 1}\label{appendix1}}
\import{paper_sections/}{appendix_1.tex}

\end{document}
